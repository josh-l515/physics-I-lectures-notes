% !TEX root = main.tex

% ---------- Geometry & Page Setup ----------
\usepackage[top=0.5cm, bottom=0.5cm, left=0.5cm, right=0.5cm]{geometry}
\setlength{\parindent}{0cm}
\setlength{\columnseprule}{0.4pt}

% ---------- Language ----------
\usepackage[spanish]{babel}

% \usepackage{fontspec} % para fuentes con XeLaTeX
% \setmainfont{Latin Modern Roman} % o cualquier otra fuente con guiones largos
%

% \setsansfont{Helvetica}
% ---------- Math ----------
\usepackage{amsmath, amssymb, amsfonts, latexsym}
\usepackage{mathrsfs} % optional: for fancy math fonts
\mathchardef\times="2202

% ---------- Figures & Layout ----------
\usepackage{graphicx}
\usepackage{float}
\usepackage{wrapfig}
\usepackage{tikz}
\usetikzlibrary{intersections}

% ---------- Tables & Columns ----------
\usepackage{array}
\usepackage{multirow}
\usepackage{multicol}

% ---------- Theorems ----------
\usepackage{ntheorem}

% ---------- Headers & Footers ----------
\usepackage{fancyhdr}

% ---------- Hyperlinks ----------
\usepackage[colorlinks=true, linkcolor=blue]{hyperref}

% ---------- Custom Environment ----------
\usepackage{xparse}
\usepackage{etoolbox}

\usepackage{xcolor}

\newcounter{excercise}



\NewDocumentEnvironment{excercise}{O{} O{0.95\linewidth} O{} m m}
{
    \vspace{1em}
    \refstepcounter{excercise}
    \noindent
    \ifstrempty{#1}
    {
        % Sin imagen
        \begin{minipage}[b]{0.97\linewidth}
            \textbf{Ejercicio \theexcercise–}%
            \phantomsection\label{#4}~#5  % Etiqueta sin salto de línea

            \ifstrempty{#3}{}{%
                \vspace{-1em}
                \begin{flushright}
                    \textbf{Resp.(s):} \textbf{#3}
                \end{flushright}
            }
        \end{minipage}
    }
    {
        % Con imagen
        \begin{minipage}[b]{0.45\linewidth}
            \textbf{Ejercicio \theexcercise–}%
            \phantomsection\label{#4}~#5

            \ifstrempty{#3}{}{%
                % \vspace{0.5em}
                \begin{flushright}
                    \textbf{Resp.(s):} \textbf{#3}
                \end{flushright}
            }
        \end{minipage}
        \hfill
        \begin{minipage}[b]{0.55\linewidth}
            \centering
            \includegraphics[width=#2]{#1}
        \end{minipage}
    }
    \vspace{0.4em}
    \hrule
}
{}


% Define Grayscale colors using the [RGB] model
% A value of 0 is Black, and 255 is White.
\definecolor{DarkGray}{RGB}{70,70,70}       % Very dark gray
\definecolor{MediumGray}{RGB}{128,128,128}   % Medium gray
\definecolor{LightGray}{RGB}{192,192,192}    % Light gray
\definecolor{ChapterGray}{RGB}{180,180,180}  % A specific gray for the chapter number
