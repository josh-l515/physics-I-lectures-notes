\documentclass[12pt]{article}
\usepackage[margin=2cm]{geometry}
\usepackage{booktabs}
\usepackage{array}
\usepackage{multirow}

\begin{document}

\section*{Rúbrica de Evaluación – Examen Práctico de Circuitos (Leyes de Kirchhoff)}

\renewcommand{\arraystretch}{1.5}
\begin{tabular}{|p{4cm}|p{3cm}|p{3cm}|p{3cm}|p{3cm}|}
\hline
\textbf{Criterio} & \textbf{Excelente \\ (4 pts)} & \textbf{Bueno \\ (3 pts)} & \textbf{Regular \\ (2 pts)} & \textbf{Deficiente \\ (1 pt)} \\
\hline
\textbf{Selección de resistencias} & Identifica correctamente todas las resistencias necesarias con sus valores exactos. & Identifica la mayoría de resistencias, con 1 error menor. & Se equivoca en más de dos valores o confunde colores. & No identifica correctamente las resistencias requeridas. \\
\hline
\textbf{Armado del circuito en protoboard} & Conexiones ordenadas, sin errores, replica fielmente el esquema. & Armado con 1 o 2 errores menores fácilmente corregibles. & Varias conexiones incorrectas que afectan el circuito. & Circuito mal armado, no corresponde al esquema. \\
\hline
\textbf{Conexión de fuentes (fem)} & Conecta todas las fuentes con la polaridad correcta y sin errores. & Conecta las fuentes con un error menor de polaridad o conexión. & Conecta con varios errores que afectan las mediciones. & No logra conectar correctamente las fuentes. \\
\hline
\textbf{Medición de voltajes} & Realiza todas las mediciones con precisión y reporta valores correctos. & Realiza la mayoría de mediciones con 1 error menor. & Varias mediciones mal realizadas o anotadas. & No mide correctamente los voltajes. \\
\hline
\textbf{Aplicación de Leyes de Kirchhoff} & Aplica correctamente las leyes en nodos y mallas, con resultados consistentes. & Aplica bien las leyes pero con algún error de cálculo menor. & Presenta errores importantes en la aplicación de las leyes. & No aplica correctamente las leyes de Kirchhoff. \\
\hline
\textbf{Presentación y orden} & Trabajo muy ordenado, claro y bien estructurado. & Presentación aceptable, con algunos detalles de orden. & Poco ordenado y confuso en algunos apartados. & Desordenado, dificulta la comprensión del trabajo. \\
\hline
\end{tabular}

\vspace{0.5cm}
\textbf{Puntaje Total:} \_\_\_/24

\end{document}
