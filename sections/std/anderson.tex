\section{Cinemática  - anderson}

La \textbf{cinemática} estudia el movimiento de los cuerpos sin considerar las causas que lo producen.  
En el caso del \textit{Movimiento Rectilíneo Uniformemente Acelerado (MRUA)}, la aceleración del cuerpo es constante, es decir, la velocidad cambia de manera uniforme en el tiempo.

\subsection{Ecuaciones del MRUA}

Las ecuaciones fundamentales que describen el movimiento son:

\begin{equation}
v = v_0 + a t
\end{equation}

\begin{equation}
x = x_0 + v_0 t + \frac{1}{2} a t^2
\end{equation}

\begin{equation}
v^2 = v_0^2 + 2a(x - x_0)
\end{equation}

Donde:
\begin{itemize}
    \item $x$ : posición final (m)
    \item $x_0$ : posición inicial (m)
    \item $v$ : velocidad final (m/s)
    \item $v_0$ : velocidad inicial (m/s)
    \item $a$ : aceleración (m/s$^2$)
    \item $t$ : tiempo transcurrido (s)
\end{itemize}

\subsection{Ejemplo}

Un cuerpo parte del reposo ($v_0 = 0$) y acelera a razón de $2 \, \text{m/s}^2$ durante $5$ segundos.  
La velocidad final será:

\begin{equation}
v = v_0 + a t = 0 + 2(5) = 10 \, \text{m/s}
\end{equation}

La distancia recorrida es:

\begin{equation}
x = x_0 + \frac{1}{2} a t^2 = 0 + \frac{1}{2}(2)(5^2) = 25 \, \text{m}
\end{equation}

Por lo tanto, el cuerpo recorre \textbf{25 metros} y alcanza una velocidad de \textbf{10 m/s}.

