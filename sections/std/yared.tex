
\section{segunda Ley de Nexton - yared}

La Segunda Ley de Newton establece que la aceleración de un objeto es directamente proporcional a la fuerza neta que actúa sobre él e inversamente proporcional a su masa. Matemáticamente, esta ley se expresa como:

\begin{equation}
    \vec{F} = m \vec{a}
\end{equation}

donde:
\begin{itemize}
    \item $\vec{F}$ es la fuerza neta aplicada al objeto (en newtons, N),
    \item $m$ es la masa del objeto (en kilogramos, kg),
    \item $\vec{a}$ es la aceleración producida (en metros por segundo al cuadrado, m/s\textsuperscript{2}).
\end{itemize}

\subsection{Aplicación}

Si sobre un objeto de masa $m = \SI{2}{kg}$ actúa una fuerza neta de $\vec{F} = \SI{10}{N}$, entonces su aceleración será:

\begin{equation}
    \vec{a} = \frac{\vec{F}}{m} = \frac{10}{2} = \SI{5}{m/s^2}
\end{equation}

\subsection{Importancia}

Esta ley es fundamental en la mecánica clásica, ya que permite predecir el movimiento de los objetos cuando se conoce la fuerza que actúa sobre ellos. Es también la base para el estudio de sistemas dinámicos más complejos.

