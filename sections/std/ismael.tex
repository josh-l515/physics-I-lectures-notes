\section{Análisis Dimensional ismael} 
\[
T = k\, l^a g^b
\]

Dimensiones:
\[
[T] = [L]^a [L][T^{-2}]^b = [L]^{a+b}[T]^{-2b}
\]

Igualamos exponentes:
\[
\begin{cases}
a + b = 0,\\
-2b = 1
\end{cases}
\Rightarrow b = -\tfrac{1}{2}, \quad a = \tfrac{1}{2}
\]
Por tanto:
\[
T = k \sqrt{\frac{l}{g}}.
\]
(\(k\) es una constante adimensional, experimentalmente \(k = 2\pi\)).

\paragraph{3. Determinación de nuevas magnitudes.}
A partir de ecuaciones conocidas, se pueden definir nuevas magnitudes.  
Ejemplo: presión \(P = \frac{F}{A} \Rightarrow [P] = [M][L^{-1}][T^{-2}]\).

\subsection{Ejemplo resuelto}
Comprobar si la ecuación \( s = ut + \tfrac{1}{2}at^2 \) es dimensionalmente correcta.

\[
[L] = [L][T^{-1}][T] + [L][T^{-2}][T^2] \Rightarrow [L] = [L] + [L].
\]
\subsection{Ejercicios propuestos}
\begin{enumerate}
    \item Verifica si la ecuación \( F = m a + v t \) es dimensionalmente homogénea.
    \item Determina la fórmula dimensional de la constante elástica \(k\) en la ley de Hooke \(F = kx\).
    \item Usando análisis dimensional, deduce cómo depende el período \(T\) de un péndulo cónico del ángulo \(\theta\), la longitud \(l\) y la gravedad \(g\).
    \item Calcula las dimensiones de la viscosidad dinámica, sabiendo que \( \eta = \frac{F}{A}\frac{dx}{dv} \).
    \item Deducir una expresión para la energía potencial gravitatoria \(E_p\) en función de la masa \(m\), la gravedad \(g\) y la altura \(h\).
\end{enumerate}
