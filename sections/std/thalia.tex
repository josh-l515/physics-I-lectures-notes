% Seccion_Dinamica_minima.tex
% Seccion_Caida_Libre.tex
\section{Caída libre - thalia}
La \emph{caída libre} es un movimiento rectilíneo uniformemente acelerado bajo la acción de la gravedad, sin resistencia del aire. La aceleración es constante y vale aproximadamente $g = 9.8\,\mathrm{m/s^2}$ hacia abajo.


Las ecuaciones del movimiento son:
\begin{align}
y &= y_0 + v_0 t - \tfrac{1}{2} g t^2, \\
v &= v_0 - g t.
\end{align}


\textbf{Ejemplo:} Si un cuerpo se deja caer desde el reposo ($v_0 = 0$) desde una altura de $h = 20\,\mathrm{m}$, el tiempo de caída es:
\begin{equation}
t = \sqrt{\frac{2h}{g}} = \sqrt{\frac{2(20)}{9.8}} \approx 2.02\,\mathrm{s}.
\end{equation}


\textbf{Ejercicio propuesto:} Un objeto se lanza hacia arriba con $v_0 = 15\,\mathrm{m/s}$. Calcula la altura máxima alcanzada.